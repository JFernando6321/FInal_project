\documentclass[a4paper,12pt]{article}
\usepackage[utf8]{inputenc}
\usepackage{amsmath}
\usepackage{graphicx}
\usepackage{natbib}
\usepackage{geometry}
\usepackage[english]{babel}
\usepackage{physics}
\usepackage{multicol}

\geometry{a4paper, margin=1in}

\title{Lennard-Jones Fluids and Monte Carlo Simulation}
\author{
    Julio Fernando Vicente Maldonado\thanks{Escuela de Ciencias Físicas y Matemáticas, Universidad de San Carlos de Guatemala} \\
     Brian David Leiva Berbena\footnotemark[1] \\
    Francisco Toledo\footnotemark[1] \\
    \texttt{joulefvicente@gmail.com, bridleiva@gmail.com, franciso@gmail.com}
}
\date{\today}

\begin{document}

\maketitle

\begin{abstract}
This report presents a detailed study of Lennard-Jones fluids using Monte Carlo simulations. The aim is to investigate the properties and behavior of these fluids under different conditions. The results are compared with theoretical predictions and previous studies to validate the simulation approach.
\end{abstract}

\section{Introduction}
Lennard-Jones fluids are a type of molecular fluid characterized by the Lennard-Jones potential. This potential is widely used in molecular simulations due to its ability to approximate the interactions between a pair of neutral atoms or molecules. Monte Carlo simulations are a computational technique used to study the thermodynamic properties of these fluids. In this report, we explore the application of Monte Carlo simulations to Lennard-Jones fluids and analyze the results.

\section{Methodology}
\subsection{Lennard-Jones Potential}
The Lennard-Jones potential is given by:
\begin{equation}
    V(r) = 4\epsilon \left[ \left( \frac{\sigma}{r} \right)^{12} - \left( \frac{\sigma}{r} \right)^{6} \right],
\end{equation}
where \( \epsilon \) is the depth of the potential well, \( \sigma \) is the finite distance at which the inter-particle potential is zero, and \( r \) is the distance between particles.

\subsection{Monte Carlo Simulation}
Monte Carlo simulations involve random sampling to obtain numerical results. For Lennard-Jones fluids, the Metropolis algorithm is typically used. The steps include initialization, configuration generation, energy calculation, and acceptance criteria based on the Boltzmann factor.

\section{Results}
The simulation results show the behavior of Lennard-Jones fluids under various temperatures and densities. The radial distribution function \( g(r) \), potential energy, and pressure are calculated and compared with theoretical predictions.

\begin{figure}[h]
    \centering
    \includegraphics[width=0.7\textwidth]{08.png}
    \caption{*Las simulaciones que obtengamos*.}
    \label{fig:gr}
\end{figure}

\section{Discussion}
The results demonstrate that Monte Carlo simulations can accurately predict the properties of Lennard-Jones fluids. The agreement with theoretical and previous simulation studies confirms the reliability of the simulation approach. 

\section{Conclusion}
Monte Carlo simulations are an effective tool for studying Lennard-Jones fluids. The findings provide insights into the thermodynamic properties and can be used to model more complex systems. 

\bibliographystyle{plainnat}
\bibliography{referencias}

\end{document}



